\documentclass{article}

\usepackage[utf8]{inputenc}
\usepackage{amsfonts}
\usepackage{amsmath}
\usepackage{graphicx}
\usepackage{indentfirst}
\usepackage{polski}

\begin{document}

\section{Distance}

To find groups of voters that vote similarly, we need to define what does it
mean for two voters to vote similarly. In our model, a vote is a subset of
projects $A \subseteq C$. We want to define a function of similarity:
\[ s : 2^C \times 2^C \to \mathbb{R} \]

Sometimes instead of similarity, it would be easier to talk about distances. It
is intuitively equivalent: voters are close if and only if they are similar.

\subsection{Different distance functions}

\subsubsection{Hamming distance}

One of intuitive distance function one could think about is the number of
projects voters disagree on -- meaning one of them approves it and the other
does not. We could define it as:
\[ d_H(A,B) = |A \Delta B| \]

In a similar fassion, we could define a similarity function as a number of
project they agree:
\[ s(A,B) = |A \cap B| \]

\subsubsection*{Lack of normalisation}

Let's consider following voters:
\begin{align*}
  A_{v_1} &= \{ a, c_1 \} \\
  A_{v_2} &= \{ b, c_1 \} \\
  A_{v_3} &= \{ a, c_1, c_2, c_3, c_4, c_5, c_6, c_7, c_8 \} \\
  A_{v_4} &= \{ b, c_1, c_2, c_3, c_4, c_5, c_6, c_7, c_8 \}
\end{align*}

Which pair of them votes more similarly: $v_1$ and $v_2$ or $v_3$ and $v_4$?

Intuitively, we would say that latter. We could explain that by saying, that
from ten projects $v_3$ and $v_4$ vote for, they agree on eight of them. In the
case of $v_1$ and $v_2$ they agree on only one of three.

However, when we look at Hamming distances, they turn out to be equal:
\[ H(A_{v_1},A_{v_2}) = |\{a,b\}| = H(A_{v_3},A_{v_4}) \]

We could construct an analogous exapmple for similarity measure that counts the
instersection.

\subsubsection{Jaccard distance}

The above example shows, we should relate number of projects voters both
approve to the number of projects they both vote for:
\[ d_J(A,B) = \frac{|A \Delta B|}{|A \cup B|} \]

By doing that we also get a nice property:
\[ d_J(A,B) \in [0, 1] \]

With it, it is easy to define Jaccard (similarity) coefficient:
\[ s_J(A,B) = 1 - d_J(A,B) = \frac{|A \cap B|}{|A \cup B|}, \]
sometimes refered to as IoU (Intersection over Union).

\subsubsection*{First axioms}

To systematize functions we consider, let's define few axioms:

\begin{itemize}
  \item Normalisation: $s(A,B),d(A,B) \in [0, 1]$
  \item Reflexivity: $s(A,A) = 1, d(A,A) = 0$
  \item Symmetry: $s(A,B) = s(B,A), d(A,B) = d(B,A)$
  \item Triangle inequality: $d(A,B) + d(B,C) \geq d(A,C)$
\end{itemize}

If a similarity (distance) function satisfies normalisation, it is easy to
define a distance corresponding to it:
\[ d(A,B) = 1 - s(A,B) \]

From distances we considered so far, Hamming distance satisfies symmetry,
but because of not being normalised, it doesn't satisfy reflexivity. Jaccard
similarity satisfies all four axioms.

\subsubsection{SimRank}

Let's consider the following projects and voters: 
\begin{align*}
  c_1 &= \text{Park near Adam Mickiewicz street} && A_{v_1} = \{c_1\} \\
  c_2 &= \text{Park near Juliusz Słowacki street} && A_{v_2} = \{c_2\} \\
  c_3 &= \text{Construction of coal-fired power plane} && A_{v_3} = \{c_3\}
\end{align*}

Intuitively we would say that $v_1$ and $v_2$ are more similar to each other
than to $v_3$. This is because projects $c_1$ and $c_2$ are more similar to
each other that to $c_3$.

This motivates the following intuition: voters are similar, if they vote on
similar projects. We now need to say what does it mean for projects to be
similar. We can do it in the same exact way: if they are approved by similar
voters. 

We can specify the above intuition saying that the similarity of two voters
(projects) is an average similarity of projects they approve (voters they are
approbed by).

So we have two functions -- voters and projects similarity:
\begin{align*}
  s^V_R : V \times V \to [0, 1] \\
  s^C_R : C \times C \to [0, 1]
\end{align*}
We define $s^V_R(v,v)=s^C_R(c,c)=1$. For $v \neq u$ and $c \neq d$
we have the following recursive equation:
\begin{align*}
  s^V_R(v, u) &= \frac{D}{|A_v| |A_u|} \sum_{c \in A_v} \sum_{d \in A_u} s^C_R(c, d) \\
  s^C_R(c, d) &= \frac{D}{|V_c| |V_d|} \sum_{v \in V_c} \sum_{u \in V_d} s^C_R(v, u)
\end{align*}
Where $V_c$ is a set of all voters that approve a project $c$, and
$D \in (0, 1)$ is a decay coefficient, which is needed for convergence of
iterative method.

This distance function satisfies normalisation, reflexivity and symmetry.
Note that reflexivity is staisfied in an artificial way -- setting it
arbitrarily in definition. For different voters, the maximal similarity they
can get is $D$.

\subsubsection*{Voters as vectors}

Before introducing next distances, let's look on voters as a vectors of length
$d = |C|$. There would be a 1 on position $i$ if a voter approves project $c_i$,
otherwise there would be a 0.

\[ \vec v = \sum_{i=1}^d \vec e_i [c_i \in A_v] \]

Note that projects a voter votes for encode a "direction" they vote in.

\subsubsection{Euclidean distance}

The most obvious distance -- when we look at voters as vectors -- is the
Euclidean distance:
\[ d_E(\vec v, \vec u) = | \vec v - \vec u | \]

It satisfies symmetry and triangle inequality, and if we additionaly divide it
by $\sqrt d$ then also normalisation and reflexivity.

\subsubsection{Cosine similarity}

A less obvious, but far more common in practice, is cosine similarity:
\[ d_C(\vec v, \vec u) = \frac{\vec v \cdot \vec u}{|\vec v||\vec u|} \]

It satisfies normalisation, reflexivity and symmetry, but does not satisfy
triangle inequality.

\subsubsection{Chord distance}

Following a intuition that "voters vote in some direction", we could normalise
the length of their vectors to 1. Then the number of projects they vote for
will not matter, only the direction will.
\[
  d_{Ch}(\vec v, \vec u) =
  \left|\frac{\vec v}{|\vec v|} - \frac{\vec u}{|\vec u|}\right|
\]

Note that
\begin{align*}
  d_{Ch}(\vec v, \vec u)^2
  &= \left|\frac{\vec v}{|\vec v|} - \frac{\vec u}{|\vec u|}\right|^2 \\
  &= \left|\frac{\vec v}{|\vec v|}\right|^2
   + \left|\frac{\vec u}{|\vec u|}\right|^2
   - 2 \left( \frac{\vec v}{|\vec v|} \cdot \frac{\vec u}{|\vec u|} \right) \\
  &= 2(1 - d_C(\vec v, \vec u))
\end{align*}

So up to a affine transformation, it is equivalent to the square root of chord
distance.

If we additionaly divide it by $\sqrt 2$, then it will statisfy normalisation,
reflexivity, symmetry and triangle inequality.

\subsection{Kernel density estimation}

To better understand the similarity coefficients, it is worth to look how do
they look on a real data.

%\begin{figure}[h]
\noindent
\includegraphics[height=\textheight]{dist_density.png}
%\end{figure}

\subsection{Correlations}

To better understand the similarity coefficients, we can examine the
correlations wetewen them.

\begin{center}
\begin{tabular}{|c|cccc|}
  \hline
    & Jaccard & Cosine & Chord & SimRank \\
  \hline
  Jaccard & 1.00 & 0.96 & 0.98 & 0.64 \\
  Cosine  & 0.96 & 1.00 & 0.97 & 0.57 \\
  Chord   & 0.98 & 0.97 & 1.00 & 0.71 \\
  SimRank & 0.64 & 0.58 & 0.71 & 1.00 \\
  \hline
\end{tabular}
\end{center}

Analogous table can be made for proejcts. We define them by switching the roles
of voters and projects. In this case we can also include projects categories
(for example: greenery, education, public space) which are sets in the data
(one project can have multiple categories). We can also include target groups
(for example: seniors, children) which are also sets. We will measure them
using Jaccard distance.

\begin{center}
\begin{tabular}{|c|cccccc|}
  \hline
    & Jaccard & Cosine & Chord & Simrank & Category & Target \\
  \hline
  Jaccard & 1.00 & 0.97 & 0.99 & 0.85 & 0.46 & 0.19 \\
  Cosine & 0.97 & 1.00 & 0.95 & 0.71 & 0.46 & 0.21 \\
  Chord & 0.99 & 0.95 & 1.00 & 0.88 & 0.45 & 0.20 \\
  Simrank & 0.85 & 0.71 & 0.88 & 1.00 & 0.35 & 0.14 \\
  Category & 0.46 & 0.46 & 0.45 & 0.35 & 1.00	& 0.23 \\
  Target & 0.19 & 0.21 & 0.20 & 0.14 & 0.23	& 1.00 \\
  \hline
\end{tabular}
\end{center}

\end{document}
